\documentclass[tikzpicture]{standalone}

\usepackage{tikz}
\usepackage{booktabs}

\usetikzlibrary{optics}

\tikzset{optics/object height=0.5cm}

\newcommand{\showcase}[1]{%
    %\def\nodename{#1}%
    \texttt{#1} & \tikz[use optics]{\node[#1] at (0,0) {};}%
}

\begin{document}
\begin{tabular}{rc}
    \toprule
    Code & Result \\\midrule
    \verb+\node[<value>] at (0,0) {};+ & Output image \\\midrule
    \showcase{lens} \\
    \showcase{lens, lens type=diverging} \\
    \showcase{lens, lens style=realistic} \\
    \showcase{lens, lens style=realistic, lens type=diverging} \\
    \showcase{slit} \\
    \showcase{double slit} \\
    \showcase{thin optics element} \\
    \showcase{polarizer} \\
    \showcase{generic optics io} \\
    %\showcase{sensor} \\ % This is listed in the styles section but not defined
    \showcase{sensor line} \\
    \showcase{mirror} \\
    \showcase{spherical mirror} \\
    \showcase{thick optics element} \\
    \showcase{heat filter} \\
    \showcase{double amici prism} \\
    \showcase{screen} \\
    \showcase{diffraction grating} \\
    \showcase{grid} \\
    \showcase{semi-transparent mirror} \\
    \showcase{diaphragm} \\
    \showcase{beam splitter} \\
    \showcase{detector} \\
    \showcase{fiber coupler} \\
    \showcase{generic lamp} \\
    \showcase{generic sensor} \\
    \showcase{halogen lamp} \\
    \showcase{spectral lamp} \\
    \showcase{laser} \\
    \showcase{laser'} \\
    \showcase{concave mirror} \\
    \showcase{convex mirror} \\
    \bottomrule
\end{tabular}

\end{document}
